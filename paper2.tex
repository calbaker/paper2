%% This is file `elsarticle-template-2-harv.tex',
%%
%% Copyright 2009 Elsevier Ltd
%%
%% This file is part of the 'Elsarticle Bundle'.
%% ---------------------------------------------
%%
%% It may be distributed under the conditions of the LaTeX Project Public
%% License, either version 1.2 of this license or (at your option) any
%% later version.  The latest version of this license is in
%%    http://www.latex-project.org/lppl.txt
%% and version 1.2 or later is part of all distributions of LaTeX
%% version 1999/12/01 or later.
%%
%% The list of all files belonging to the 'Elsarticle Bundle' is
%% given in the file `manifest.txt'.
%%
%% Template article for Elsevier's document class `elsarticle'
%% with harvard style bibliographic references
%%
%% $Id: elsarticle-template-2-harv.tex 155 2009-10-08 05:35:05Z rishi $
%% $URL: http://lenova.river-valley.com/svn/elsbst/trunk/elsarticle-template-2-harv.tex $
%%
\documentclass[preprint,authoryear,12pt]{elsarticle}
% \documentclass{article}

%% Use the option review to obtain double line spacing
%% \documentclass[authoryear,preprint,review,12pt]{elsarticle}

%% Use the options 1p,twocolumn; 3p; 3p,twocolumn; 5p; or 5p,twocolumn
%% for a journal layout:
%% \documentclass[final,authoryear,1p,times]{elsarticle}
%% \documentclass[final,authoryear,1p,times,twocolumn]{elsarticle}
%% \documentclass[final,authoryear,3p,times]{elsarticle}
%% \documentclass[final,authoryear,3p,times,twocolumn]{elsarticle}
%% \documentclass[final,authoryear,5p,times]{elsarticle}
%% \documentclass[final,authoryear,5p,times,twocolumn]{elsarticle}

%% if you use PostScript figures in your article
%% use the graphics package for simple commands
%% \usepackage{graphics}
%% or use the graphicx package for more complicated commands
%% \usepackage{graphicx}
%% or use the epsfig package if you prefer to use the old commands
%% \usepackage{epsfig}

%% stuff inserted by CB
\usepackage[latin1]{inputenc}
\usepackage{amsmath}
\usepackage{amsfonts}
\usepackage{amssymb}
\usepackage{amsthm}
\usepackage{fancyhdr}
\usepackage{geometry}
\usepackage{color}
\usepackage{natbib}
\usepackage{grffile} % allows for spaces in figure filenames
\usepackage{graphicx}
\usepackage{lipsum}
\usepackage{float}
\usepackage[bf]{caption}
\usepackage{multirow} % enables "merging" in tabular environment
\usepackage[table]{xcolor} % enables coloring of cells in tables
\usepackage{subfig} % allows use of subfigures
\usepackage{textcomp} % allows use of degree symbol
\usepackage{mathrsfs}
\usepackage{nomencl}
\usepackage{chngcntr}
\usepackage{hyperref}

%% The amssymb package provides various useful mathematical symbols
\usepackage{amssymb}
%% The amsthm package provides extended theorem environments
%% \usepackage{amsthm}

\graphicspath{{./figures/}}

%% The lineno packages adds line numbers. Start line numbering with
%% \begin{linenumbers}, end it with \end{linenumbers}. Or switch it on
%% for the whole article with \linenumbers after \end{frontmatter}.
%% \usepackage{lineno}

%% natbib.sty is loaded by default. However, natbib options can be
%% provided with \biboptions{...} command. Following options are
%% valid:

%%   round  -  round parentheses are used (default)
%%   square -  square brackets are used   [option]
%%   curly  -  curly braces are used      {option}
%%   angle  -  angle brackets are used    <option>
%%   semicolon  -  multiple citations separated by semi-colon (default)
%%   colon  - same as semicolon, an earlier confusion
%%   comma  -  separated by comma
%%   authoryear - selects author-year citations (default)
%%   numbers-  selects numerical citations
%%   super  -  numerical citations as superscripts
%%   sort   -  sorts multiple citations according to order in ref. list
%%   sort&compress   -  like sort, but also compresses numerical citations
%%   compress - compresses without sorting
%%   longnamesfirst  -  makes first citation full author list
%%
%% \biboptions{longnamesfirst,comma}

% \biboptions{}

\journal{undecided}

\begin{document}

\begin{frontmatter}

%% Title, authors and addresses

%% use the tnoteref command within \title for footnotes;
%% use the tnotetext command for the associated footnote;
%% use the fnref command within \author or \address for footnotes;
%% use the fntext command for the associated footnote;
%% use the corref command within \author for corresponding author footnotes;
%% use the cortext command for the associated footnote;
%% use the ead command for the email address,
%% and the form \ead[url] for the home page:
%%
%% \title{Title\tnoteref{label1}}
%% \tnotetext[label1]{}
%% \author{Name\corref{cor1}\fnref{label2}}
%% \ead{email address}
%% \ead[url]{home page}
%% \fntext[label2]{}
%% \cortext[cor1]{}
%% \address{Address\fnref{label3}}
%% \fntext[label3]{}

  \title{High Fidelity Finite Difference Model for Exploring Multi-DOF Thermoelectric Generator Design Space}

%% use optional labels to link authors explicitly to addresses:
%% \author[label1,label2]{<author name>}
%% \address[label1]{<address>}
%% \address[label2]{<address>}

\author[UT]{Chad A. Baker}  
\ead{calbaker@utexas.edu}
\author[UT]{Haiyan Fateh}  
\author[UT]{Li Shi}  
\author[UT]{Matthew J. Hall}

\address[UT]{Mechanical Engineering \\ University of Texas \\ 1 University
  Station, C2200 \\ Austin, TX 78712} 

\begin{abstract}
%% Text of abstract
  stuff
\end{abstract}

\begin{keyword}
%% keywords here, in the form: keyword \sep keyword
thermoelectric \sep model

%% MSC codes here, in the form: \MSC code \sep code
%% or \MSC[2008] code \sep code (2000 is the default)

\end{keyword}

\end{frontmatter}

% \linenumbers

%% main text

% \makenomenclature
% \printnomenclature[1in]

\section{Introduction}
\label{sec:intro}

% Chad + Haiyan

One third or more of the energy content of the fuel of a typical
diesel engine is expelled through the tailpipe as waste heat, so any
amount of waste heat usefully recovered is a benefit to vehicle
efficiency \cite{heywood_internal_1988,caton_operating_2000,%
  endo_study_2007}. Potential waste heat recovery technologies
currently under consideration for automotive applications are organic
Rankine cycles \cite{miller_modeling_2009}, turbo-compounding
\cite{weerasinghe_thermal_2010}, direct usage of the waste heat as
thermal energy, powering absorption chillers \cite{talom_heat_2009},
and thermoelectric (TE) devices
\cite{miller_modeling_2009,hendricks_thermal_2007,hendricks_advanced_2002%
  ,hussain_thermoelectric_2009,crane_introduction_2011,%
  crane_optimization_2004,crane_towards_2001}.  Thermoelectric devices
in particular are appealing for automotive applications in that they
do not have moving parts, and thermal energy can be converted into
electricity directly.  Their high cost and low efficiency are
currently issues, but the potential use of relatively inexpensive and
abundant element materials such as Mg$_2$Si$_{0.5}$Sn$_{0.5}$ and
MnSi$_{1.75}$ for TE devices shows promise for future cost reductions
and improved performance \cite{rowe_thermoelectrics_2006}.  A
schematic of a TE leg pair, which is the most fundamental part of a TE
device (a device typically consists of hundreds of leg pairs) is shown
in Figure \ref{fig:te-pair-first-look}.
\begin{figure}[H]
  \centering
  \includegraphics[width=0.8\textwidth]{te-leg-pair}
  \caption{Schematic of TE leg pair.  A TE leg pair is the smallest
    unit that can function as a standalone TE device.  A typical TE
    device consists of hundreds of these pairs in series and/or
    parallel arrangement.}
  \label{fig:te-pair-first-look}
\end{figure}

A number of researchers have reported both modeling and experimental
work with TE vehicle waste heat recovery with varying emphasis on heat
exchanger performance and TE device performance. Stobart and coworkers
modeled and experimentally tested TE performance for devices with
exhaust temperatures up to 800 K.  Because of internal thermoelectric
conversion, there is a difference between the hot side and cold side
heat transfer of TE devices. This heat transfer asymmetry was not
accounted for in the work by Stobart and coworkers.  They used an
efficiency model based on the average figure of merit ($ZT$) of the TE
material that assumes optimal device geometry and optimal current
\cite{stobart_potential_2009,stobart_potential_2010}.  Optimal
geometry consists of n-/p-type leg area ratio, leg height, the
distance between adjacent legs. Modeling efforts by Hendricks \emph{et
  al.}  \cite{hendricks_advanced_2002,hendricks_thermal_2007} reported
using temperature-dependent TE properties (Seebeck coefficient,
electrical resistivity, and thermal conductivity) to determine optimal
TE leg areas, lengths, and device designs. Miller \emph{et al.}
studied heat transfer and heat exchanger optimization for a combined
TE and organic Rankine cycle waste heat recovery system. Miller
\emph{et al.}  calculated the TE device efficiency based on an average
$ZT$ for state of the art TE materials evaluated at typical operating
temperatures \cite{miller_modeling_2009}.  Hussain \emph{et al.}
developed a model with single node TE devices that accounted for
transient behavior and thermal asymmetry.  In their model, the spatial
variation and temperature dependence of the TE properties in
individual TE devices were not accounted for.  No attempt was made to
account for restriction of the exhaust flow in the form of back
pressure on the engine \cite{hussain_thermoelectric_2009}. Back
pressure is an important design consideration because the pumping
effort the engine must exert to expel exhaust gases is increased if
back pressure in the exhaust system is increased.  Increased pumping
effort takes useful power away from the engine crank shaft.

Crane and coworkers developed a model for cross-flow and counter-flow
heat exchanger configurations.  They used an analytical,
thermally-lumped TE leg performance model that correctly accounted for
thermal asymmetry.  Some of this modeling work was also validated with
experimental results. Crane's most recent model incorporated transient
performance \cite{crane_towards_2001,crane_optimization_2004,%
  crane_introduction_2011}. None of the previous work discussed here
used a TE model that accounted for spatial- and temperature-variant
properties within the TE material of an individual TE couple.

This dissertation reports a modeling study of TE vehicle waste heat
recovery devices based on Mg$_2$Si$_{0.5}$Sn$_{0.5}$ and MnSi$_{1.75}$
materials with consideration of both system level heat exchanger
performance and TE device performance.  The model presented in this
dissertation offers an improvement over existing models because it
used a finite difference method to account for spatial- and
temperature-variant TE properties at the individual TE leg pair
scale. The model also coupled the hot- and cold-side convection heat
flux to the TE heat flux, thus accounting for the thermal asymmetry,
using a numerical root finding algorithm.  This model incorporated
back pressure as part of an optimization metric in the form of pumping
power requirement.  The output of the system model was heat transfer,
TE power, pumping power requirement, and the net power output, where
the net power output is the total TE power minus the pumping power
required to move the coolant and exhaust through the heat
exchanger. The model was used to predict overall performance of
several heat exchanger systems, incorporating different fin geometries
and flow arrangements for enhancing heat transfer and TE power
generation.  The important TE parameters were optimized using a
numerical algorithm, and the optimization metric used by the algorithm
was the net power of the system.

This work also includes two generations of TE waste heat recovery
systems that were built and tested in the exhaust system of a Cummins
6.7 L turbo Diesel engine. The experimental work was used to validate
the model so that the model could be used both as a design tool and a
means of thoroughly optimizing various parameters in the TE waste heat
recovery system design space.

The first generation TE system consisted of a compact heat exchanger
without any TE devices.  This system was built to validate the
convection heat transfer and flow model using only a small portion of
the exhaust flow of the Cummins engine.  In order to validate the
model at full scale, a second generation TE heat exchanger system was
constructed to utilize the entire flow of the Cummins engine.
Experimental apparatus, methods, and results from each heat exchanger
will be presented separately in Chapters \ref{cha:te-first-generation}
and \ref{cha:te-second-generation}.  

A key contribution of this work was recognizing that for almost any
practical TE device application, the boundary conditions on the hot
and cold side will be convection boundary conditions rather than
specified temperature boundary conditions.  The TE model developed
here accounts for that, and in addition, the model accounts for
spatially/thermally variant properties in the direction of heat flow
and thermal asymmetry.

\section{TE Device Testing}
\label{sec:te-device-testing}

% Haiyan is in charge of this section

This may be beyond the scope.  

\subsection{Steady State}
\label{sec:exp-steady-state}

This may be beyond the scope.  

\subsection{Transient}
\label{sec:exp-transient}

This may be beyond the scope.  

\section{Model}
\label{sec:model}

\subsection{Analytical Model}
\label{sec:analy-model}

% Chad will handle this section

The TE devices were modeled using Domenicali's energy balance equation
and the thermoelectric heat flux equation
\cite{domenicali_irreversible_1953,hogan_modeling_2006}:
\begin{equation}
  \label{eq:te-energy-balance}
  \frac{d}{d x} \left(k \frac{d T}{d x}
  \right) = - \rho J^2 + J T \frac{d \alpha}{d x}
\end{equation}
and
\begin{equation}
  \label{eq:te-heat-flux}
  q = J T \alpha - k \frac{d T}{d x}
\end{equation}
where $x$ is the coordinate along the direction of heat and current
flux, $k$ is the thermal conductivity, $T$ is the temperature, $\rho$
is the electrical resistivity, $J$ is the current flux, and $\alpha$
is the Seebeck coefficient, all evaluated at $T$.

For completeness, the entire solution for Equations
\ref{eq:te-energy-balance} and \ref{eq:te-heat-flux} will be presented
here.  This solution has been presented in the literature
\cite{hogan_modeling_2006} with a critical sign error, and the present
work corrects this error.

A tractable solution can be achieved through several steps of
algebraic manipulation.  First, rearranging Equation
\ref{eq:te-heat-flux} yields
\begin{equation}
  \label{eq:te-temp-gradient-final}
  \frac{dT}{dx} = \frac{J T \alpha -q}{k}
\end{equation}
This is the final equation for determining the temperature
gradient. In order to determine the heat flux gradient, Equation
\ref{eq:te-temp-gradient-final} is multiplied by $k$ and substituted
into Equation \ref{eq:te-energy-balance}, resulting in
\begin{equation}
  \label{eq:te-energy-step1}
  \frac{d}{dx} \left( J T \alpha - q \right) = - \rho J^2 + J T \frac{d
    \alpha}{d x}
\end{equation}
For constant current density (e.g. if the cross-sectional area of the
leg is constant), the product rule can be used to expand the
left-hand-side to
\begin{equation}
  \label{eq:te-energy-step2}
  J \left( \frac{dT}{dx} \alpha + \frac{d \alpha}{dx} T \right) -
  \frac{dq}{dx} = -
  \rho J^2 + J T \frac{d \alpha}{d x}  
\end{equation}
The temperature gradient in first term on the left-hand-side of
Equation \ref{eq:te-energy-step2} can be replaced by substituting
Equation \ref{eq:te-temp-gradient-final}, and after some rearranging,
this yields
\begin{equation}
  \label{eq:te-energy-step3}
  \frac{dq}{dx} = \rho J^2 + J \alpha \frac{J T \alpha -q}{k}
\end{equation}
Multiplying the second term ($J \alpha \frac{J T \alpha -q}{k}$) by
$\frac{\rho}{\rho}$ and recognizing the definition of $ZT$ (Equation
\ref{eq:ZT-def}) yields the final result:
\begin{equation}
  \label{eq:te-energy-final}
  \frac{dq}{dx} = \rho J^2 \left( 1 + ZT \right) - \frac{J \alpha
    q}{k} 
\end{equation}

Current density, rather than load resistance, is specified in the
model.  However, it may be desired to specify load resistance for
experimental or design purposes.  To fully close the model for a
specified load resistance, an expression is needed to relate current
density to load resistance.  This is given by
\begin{equation}
  \label{eq:current-v-R_L}
  R_{\text{L}} = \frac{V_{\text{S,n}} - I / A_{\text{n}} \rho_{\text{n}} l +
    V_{\text{S,p}} - I / A_{\text{p}} \rho_{\text{p}} l}{I} 
\end{equation}
where $R_{\text{L}}$ is the load electrical resistance, $l$ is the
length of the TE legs, subscripts n and p indicate n- or p-type leg,
and $V_{\text{S}}$ is the Seebeck voltage, given by
\begin{equation}
  \label{eq:V_Seebeck}
  V_{\text{S}} = \sum_{i=1}^N \alpha_i \left( T_i - T_{i-1} \right)
\end{equation}%
and the current, $I$, is given by
\begin{equation}
  \label{eq:current-flux}
  I = J A
\end{equation}
where $J$ is current density of either leg, and $A$ is the
cross-sectional area of either leg.  Either leg can be used for this
calculation, provided the same leg is used for both the current
density and the area. The total power was calculated by
\begin{equation}
  \label{eq:electrical-power}
  \dot{W}_{\text{elec}} = I R_{\text{L}}
\end{equation}
where $\dot{W}_{\text{elec}}$ is the electrical power output.

For a pair of thermoelectric legs with convection heat transfer on
both sides and isothermal interconnects with no generation, the
boundary conditions are combined heat flux and temperature.  The
interconnects, which are typically copper, are assumed to be
isothermal because the thermal conductivity is high. Generation due to
Joule heating is assumed to be negligible due to high electrical
conductivity, and thermoelectric energy conversion at the interface of
the interconnects and the TE materials is neglected.  A schematic that
will be referenced for both the analytical and numerical finite
difference equations is shown in Figure
\ref{fig:te-numerical-schematic}.  On both the hot and cold side of
the TE leg pair, the temperature of each leg must be equal to the
temperature of the isothermal interconnect interface, and this is also
the temperature driving convection heat flux.  The composite heat flux
is given by
\begin{equation}
  \label{eq:q-composite-definition}
  q_{\text{comp}} = \frac{A_\text{n} q_\text{n} + A_\text{p}
    q_\text{p}}{A_\text{n} + A_\text{p} + A_{\text{void}}} 
\end{equation}
where $A$ is the cross-sectional area of the n- or p-type leg or that
of the void space, and $q$ is the composite heat flux or the heat flux
in the specified leg.  Subscripts n, p, void, and comp indicate
whether the variable references the n-type, p-type, void space, or
composite value.  A visual representation of the p-type, n-type and
void areas is shown in Figure \ref{fig:te-area-visual}.  The void area
is the empty space between adjacent TE legs, and it is assumed to be
perfectly insulated.
\begin{figure}[H]
  \centering
  \includegraphics[width=0.5\textwidth]{te-area-schematic}
  \caption{Visual reprentation of p-type, n-type, and void areas.}
  \label{fig:te-area-visual}
\end{figure}
\noindent%
The composite heat flux must be equal to the convection heat flux on
both the hot and cold sides, given by
\begin{equation}
  \label{eq:conv-flux-hot}
  q_\text{h,comp} = U_\text{h} \left( T_{\text{h,conv}} - T_\text{h}
  \right)   
\end{equation}
for the hot side and
\begin{equation}
  \label{eq:conv-flux-cold}
  q_\text{c,comp} = U_\text{c} \left( T_{\text{c,conv}} - T_\text{c}
  \right)   
\end{equation}
for the cold side, where $U$ is the overall heat transfer coefficient
for the hot or cold side (including thermal resistance due to
conduction and contact resistances associated with the Al plate,
substrate, and interconnect shown in Figure
\ref{fig:te-numerical-schematic} as well as thermal resistance due to
convection; see Section \ref{sec:te-parasitic-losses}), $T_{conv}$ is
the temperature of the hot or cold fluid, $T$ is the temperature of
the hot or cold side of the TE leg pair, and $q_{\text{comp}}$ is the
composite heat flux on the hot or cold side of the TE device, given in
Equation \ref{eq:q-composite-definition}.  Subscripts $h$ and $c$
indicate whether the variable corresponds to the hot or cold side.
The temperature boundary conditions are specified as follows:
\begin{equation}
  \label{eq:TE-hot-ana-T}
  T_{\text{n,h}} = T_{\text{p,h}} = T_\text{h}
\end{equation}
\begin{equation}
  \label{eq:TE-cold-ana-T}
  T_{\text{n,c}} = T_{\text{p,c}} = T_\text{c}
\end{equation}
where, as before, subscripts n and p indicate n- or p-type leg, and
subscripts c and h indicate cold or hot side.

\subsection{Thermoelectric Finite Difference Model}
\label{sec:te_finite}

Following the procedure from Hogan and Shih
\cite{hogan_modeling_2006}, the equations from Section
\ref{sec:analy-model} can be discretized to the following pair of
first order, reverse-looking finite difference equations:
\begin{equation}
  \label{eq:temp-discrete}
  T_i = T_{i-1} + \left( \frac{J T_{i-1} \alpha_{i-1} -q_{i-1}}{k
     _{i-1}} \right) \Delta x  
\end{equation}
from Equation \ref{eq:te-temp-gradient-final}, and
\begin{equation}
  \label{eq:q-discrete}
  q_i = q_{i-1} + \left( \rho_{i-1} J^2 + \left( 1 + ZT_{i-1} \right) - \frac{J
      \alpha_{i-1} q_{i-1}}{k_{i-1}} \right) \Delta x
\end{equation}
from Equation \ref{eq:te-energy-final}, where the subscripts $_i$
and $_{i-1}$ indicate the current and previous nodes, respectively,
and the TE properties are all evaluated at the temperature of the
previous node.  A schematic of the system being discretized is shown
in Figure \ref{fig:te-numerical-schematic}.
\begin{figure}[H]
  \centering
  \includegraphics[width=0.75\textwidth]{te-numerical-schematic}
  \caption{Schematic of numerical scheme used to solve TE leg pair.
    Blue leg is n-type and red leg is p-type.}
  \label{fig:te-numerical-schematic}
\end{figure}

The algorithm used to solve these coupled finite difference equations
deviated substantially from the method described by Hogan and Shih
\cite{hogan_modeling_2006} in that the algorithm accounted for a
convection (Neumann) boundary condition rather than a temperature
(Dirichlet) boundary condition.  This is important because in most
practical applications will have temperature-driven heat flux, or
convection, boundary conditions.  The necessary boundary conditions
are fluid temperature and heat transfer coefficient on both the hot
and cold side of the TE device as well as the hot and cold side
temperature of the TE device.  The hot and cold side TE device
temperature boundary conditions are satisfied by
\begin{equation}
  \label{eq:te-hot-temp-bc}
  T_{0,\text{n}} = T_{0,\text{p}}
\end{equation}
and
\begin{equation}
  \label{eq:te-cold-temp-bc}
  T_{N,\text{n}} = T_{N,\text{p}}
\end{equation}
where subscript 0 indicates the node in contact with the hot side
copper interconnect, subscript $N$ indicates the node in contact with
the cold side copper interconnect, and the second subscript indicates
n-type or p-type TE leg. The heat flux boundary conditions are given
by following conditions:
\begin{equation}
  \label{eq:te-hot-side-BC}
  U_{\text{h}} \left( T_{\text{h,conv}} - T_{\text{h}} \right) =
  q_{\text{h,comp}} 
\end{equation}
and
\begin{equation}
  \label{eq:te-cold-side-BC}
  U_{\text{c}} \left( T_{\text{c,conv}} - T_{\text{c}} \right) =
  q_{\text{c,comp}} 
\end{equation}

The algorithm proceeds as follows:
\begin{enumerate}
\item Estimate hot side heat flux for the n-type leg, hot side heat
  flux for the p-type leg, and hot side temperature based on pure
  conduction and convection.
\item Solve for the temperature and heat flux profiles in both the
  n-type and p-type TE legs using Equations \ref{eq:temp-discrete}
  and \ref{eq:q-discrete}.
\item Determine the error between the composite TE heat flux and the
  convection heat flux (Equations \ref{eq:te-hot-side-BC} and
  \ref{eq:te-cold-side-BC}) \label{item:heat-flux-bc} for both the hot
  and cold sides.
\item Determine the error between the cold side p-type and n-type
  temperatures, as given by Equations \ref{eq:te-hot-temp-bc} and
  \ref{eq:te-cold-temp-bc}.  \label{item:cold-temperature-bc}
\item Iterate until the error in steps \ref{item:heat-flux-bc} and
  \ref{item:cold-temperature-bc} is zero.
\end{enumerate}
The iteration was performed by the \emph{fsolve} function of the open
source Python 2.7 SciPy package.

\subsection{Steady State}
\label{sec:mod-steady-state}

\subsubsection{Design Space and Optimization}
\label{sec:mod-design-space}

\subsection{Transient}
\label{sec:mod-transient}

This may be beyond the scope.  

\section{Results and Discussion}
\label{sec:RandD}

% Chad + Haiyan

\subsection{Steady State}
\label{sec:r-d-steady-state}

% Chad + Haiyan

\subsubsection{Design Space and Optimization}
\label{sec:r-d-design-space}

% Chad

As indicated in Section \ref{sec:te-system-model}, n-/p-type leg area
ratio, leg length, fill fraction, and current (as a surrogate for load
resistance) were all used as optimization parameters for maximizing
net power output.  Optimal n-/p-type leg area ratio remained
relatively constant with respect to the other parameters.  This is
because temperature dependency of the thermoelectric properties
exhibited similar trends for both the n- and p-type materials.  The
latter three parameters, leg length, fill fraction, and current,
exhibited some strongly interdependent tendencies in the way they
affected power output, and to demonstrate this, surface projections of
the 3-dimensional design space are plotted in Figure
\ref{fig:te_design_space}. The optimal values for each parameter in
the design space are shown in Table \ref{tab:te-design-space-opt}.
For this design space, n-/p-type leg area ratio is held constant at
0.7 because there is almost no change in optimal leg area as the other
parameters are varied.  Convection boundary conditions are 300 K and
680 K with overall heat transfer coefficients of 8
$\frac{\text{kW}}{\text{m}^2\text{K}}$ and 2
$\frac{\text{kW}}{\text{m}^2\text{K}}$ for the coolant and exhaust,
respectively.  These boundary condition are typical values for coolant
and exhaust averaged in the stream-wise direction in the model heat
exchangers.

\begin{figure}[H]
  \centering
  \subfloat[]{\label{fig:te-power-space2}%
    \includegraphics[width=0.5\textwidth]{power_I_fill}}
  \subfloat[]{\label{fig:te-power-space3}%
    \includegraphics[width=0.5\textwidth]{power_length_I}} \\
  \subfloat[]{\label{fig:te-power-space1}%
    \includegraphics[width=0.5\textwidth]{power_fill_length}}
  \caption{3D Surface plots showing thermoelectric power output v. (a)
    current and fill fraction, (b) current and length, and (c) fill
    fraction and length.  n-/p-type leg area ratio is held constant at
    0.7.  Convection boundary conditions are 300 K and 680 K with
    overall heat transfer coefficients of 8
    $\frac{\text{kW}}{\text{m}^2\text{K}}$ and 2
    $\frac{\text{kW}}{\text{m}^2\text{K}}$ for the coolant and
    exhaust, respectively.}  % give these conditions in the text
  \label{fig:te_design_space}
\end{figure}
\begin{table}[H]
  \centering
  \caption{Optimal parameters for design space of standalone TE
    device, as determined by model. Convection boundary conditions are
    300 K and 680 K with overall heat transfer coefficients of 8
    $\frac{\text{kW}}{\text{m}^2\text{K}}$ and 2
    $\frac{\text{kW}}{\text{m}^2\text{K}}$ for the coolant and
    exhaust, respectively.}
  \begin{tabular}[H]{cc}
    Parameter & Value \\
    \hline
    fill fraction & 17.9 \% \\
    leg length & 0.330 mm\\
    current & 23.3 A \\
    n-/p-type leg area ratio & 0.700 \\
  \end{tabular}
  \label{tab:te-design-space-opt}
\end{table}
These results indicate that changing any of the three variables (fill
fraction, leg length, or current) can greatly impact the effect the
other two have on performance.  This also shows that there is a
clearly defined optimal location within the design space.  Leg area
ratio is included because it does require optimization, but the
optimal n-/p-type leg area ratio is insensitive to the other three
parameters as well as boundary conditions.

Note that the results presented in Figure \ref{fig:te_design_space}
vary with convection boundary conditions, and as such, the HX system
model would produce results that are somewhat shifted in the design
space due to a range of convection boundary conditions throughout the
stream-wise direction of the heat exchanger.  Exploring this
relationship is beyond the scope of the present work.  

\subsection{Transient}
\label{sec:r-d-transient}

% Haiyan 
This may be beyond the scope.  

\section{Conclusions}
\label{sec:conclusions}

% Chad + Haiyan

\section*{Acknowledgments}
\label{sec:acknowledgements}

%% The Appendices part is started with the command \appendix;
%% appendix sections are then done as normal sections
%% \appendix

%% \section{}
%% \label{}

%% References
%%
%% Following citation commands can be used in the body text:
%%
%%  \citet{key}  ==>>  Jones et al. (1990)
%%  \citep{key}  ==>>  (Jones et al., 1990)
%%
%% Multiple citations as normal:
%% \citep{key1,key2}         ==>> (Jones et al., 1990; Smith, 1989)
%%                            or  (Jones et al., 1990, 1991)
%%                            or  (Jones et al., 1990a,b)
%% \cite{key} is the equivalent of \citet{key} in author-year mode
%%
%% Full author lists may be forced with \citet* or \citep*, e.g.
%%   \citep*{key}            ==>> (Jones, Baker, and Williams, 1990)
%%
%% Optional notes as:
%%   \citep[chap. 2]{key}    ==>> (Jones et al., 1990, chap. 2)
%%   \citep[e.g.,][]{key}    ==>> (e.g., Jones et al., 1990)
%%   \citep[see][pg. 34]{key}==>> (see Jones et al., 1990, pg. 34)
%%  (Note: in standard LaTeX, only one note is allowed, after the ref.
%%   Here, one note is like the standard, two make pre- and post-notes.)
%%
%%   \citealt{key}          ==>> Jones et al. 1990
%%   \citealt*{key}         ==>> Jones, Baker, and Williams 1990
%%   \citealp{key}          ==>> Jones et al., 1990
%%   \citealp*{key}         ==>> Jones, Baker, and Williams, 1990
%%
%% Additional citation possibilities
%%   \citeauthor{key}       ==>> Jones et al.
%%   \citeauthor*{key}      ==>> Jones, Baker, and Williams
%%   \citeyear{key}         ==>> 1990
%%   \citeyearpar{key}      ==>> (1990)
%%   \citetext{priv. comm.} ==>> (priv. comm.)
%%   \citenum{key}          ==>> 11 [non-superscripted]
%% Note: full author lists depends on whether the bib style supports them;
%%       if not, the abbreviated list is printed even when full requested.
%%
%% For names like della Robbia at the start of a sentence, use
%%   \Citet{dRob98}         ==>> Della Robbia (1998)
%%   \Citep{dRob98}         ==>> (Della Robbia, 1998)
%%   \Citeauthor{dRob98}    ==>> Della Robbia


%% References with bibTeX database:

\newpage
\bibliographystyle{model2-names}
\bibliography{Dissertation}

%% Authors are advised to submit their bibtex database files. They are
%% requested to list a bibtex style file in the manuscript if they do
%% not want to use model2-names.bst.

%% References without bibTeX database:

% \begin{thebibliography}{00}

%% \bibitem must have one of the following forms:
%%   \bibitem[Jones et al.(1990)]{key}...
%%   \bibitem[Jones et al.(1990)Jones, Baker, and Williams]{key}...
%%   \bibitem[Jones et al., 1990]{key}...
%%   \bibitem[\protect\citeauthoryear{Jones, Baker, and Williams}{Jones
%%       et al.}{1990}]{key}...
%%   \bibitem[\protect\citeauthoryear{Jones et al.}{1990}]{key}...
%%   \bibitem[\protect\astroncite{Jones et al.}{1990}]{key}...
%%   \bibitem[\protect\citename{Jones et al., }1990]{key}...
%%   \harvarditem[Jones et al.]{Jones, Baker, and Williams}{1990}{key}...
%%

% \bibitem[ ()]{}

% \end{thebibliography}

\end{document}

%%
%% End of file `elsarticle-template-2-harv.tex'.

%%% Local Variables: 
%%% mode: latex
%%% TeX-master: t
%%% End: 
